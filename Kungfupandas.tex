% Options for packages loaded elsewhere
\PassOptionsToPackage{unicode}{hyperref}
\PassOptionsToPackage{hyphens}{url}
%
\documentclass[
]{book}
\usepackage{amsmath,amssymb}
\usepackage{lmodern}
\usepackage{iftex}
\ifPDFTeX
  \usepackage[T1]{fontenc}
  \usepackage[utf8]{inputenc}
  \usepackage{textcomp} % provide euro and other symbols
\else % if luatex or xetex
  \usepackage{unicode-math}
  \defaultfontfeatures{Scale=MatchLowercase}
  \defaultfontfeatures[\rmfamily]{Ligatures=TeX,Scale=1}
\fi
% Use upquote if available, for straight quotes in verbatim environments
\IfFileExists{upquote.sty}{\usepackage{upquote}}{}
\IfFileExists{microtype.sty}{% use microtype if available
  \usepackage[]{microtype}
  \UseMicrotypeSet[protrusion]{basicmath} % disable protrusion for tt fonts
}{}
\makeatletter
\@ifundefined{KOMAClassName}{% if non-KOMA class
  \IfFileExists{parskip.sty}{%
    \usepackage{parskip}
  }{% else
    \setlength{\parindent}{0pt}
    \setlength{\parskip}{6pt plus 2pt minus 1pt}}
}{% if KOMA class
  \KOMAoptions{parskip=half}}
\makeatother
\usepackage{xcolor}
\IfFileExists{xurl.sty}{\usepackage{xurl}}{} % add URL line breaks if available
\IfFileExists{bookmark.sty}{\usepackage{bookmark}}{\usepackage{hyperref}}
\hypersetup{
  pdftitle={Kungfu Pandas},
  pdfauthor={Lê Huỳnh Đức},
  hidelinks,
  pdfcreator={LaTeX via pandoc}}
\urlstyle{same} % disable monospaced font for URLs
\usepackage{color}
\usepackage{fancyvrb}
\newcommand{\VerbBar}{|}
\newcommand{\VERB}{\Verb[commandchars=\\\{\}]}
\DefineVerbatimEnvironment{Highlighting}{Verbatim}{commandchars=\\\{\}}
% Add ',fontsize=\small' for more characters per line
\usepackage{framed}
\definecolor{shadecolor}{RGB}{248,248,248}
\newenvironment{Shaded}{\begin{snugshade}}{\end{snugshade}}
\newcommand{\AlertTok}[1]{\textcolor[rgb]{0.94,0.16,0.16}{#1}}
\newcommand{\AnnotationTok}[1]{\textcolor[rgb]{0.56,0.35,0.01}{\textbf{\textit{#1}}}}
\newcommand{\AttributeTok}[1]{\textcolor[rgb]{0.77,0.63,0.00}{#1}}
\newcommand{\BaseNTok}[1]{\textcolor[rgb]{0.00,0.00,0.81}{#1}}
\newcommand{\BuiltInTok}[1]{#1}
\newcommand{\CharTok}[1]{\textcolor[rgb]{0.31,0.60,0.02}{#1}}
\newcommand{\CommentTok}[1]{\textcolor[rgb]{0.56,0.35,0.01}{\textit{#1}}}
\newcommand{\CommentVarTok}[1]{\textcolor[rgb]{0.56,0.35,0.01}{\textbf{\textit{#1}}}}
\newcommand{\ConstantTok}[1]{\textcolor[rgb]{0.00,0.00,0.00}{#1}}
\newcommand{\ControlFlowTok}[1]{\textcolor[rgb]{0.13,0.29,0.53}{\textbf{#1}}}
\newcommand{\DataTypeTok}[1]{\textcolor[rgb]{0.13,0.29,0.53}{#1}}
\newcommand{\DecValTok}[1]{\textcolor[rgb]{0.00,0.00,0.81}{#1}}
\newcommand{\DocumentationTok}[1]{\textcolor[rgb]{0.56,0.35,0.01}{\textbf{\textit{#1}}}}
\newcommand{\ErrorTok}[1]{\textcolor[rgb]{0.64,0.00,0.00}{\textbf{#1}}}
\newcommand{\ExtensionTok}[1]{#1}
\newcommand{\FloatTok}[1]{\textcolor[rgb]{0.00,0.00,0.81}{#1}}
\newcommand{\FunctionTok}[1]{\textcolor[rgb]{0.00,0.00,0.00}{#1}}
\newcommand{\ImportTok}[1]{#1}
\newcommand{\InformationTok}[1]{\textcolor[rgb]{0.56,0.35,0.01}{\textbf{\textit{#1}}}}
\newcommand{\KeywordTok}[1]{\textcolor[rgb]{0.13,0.29,0.53}{\textbf{#1}}}
\newcommand{\NormalTok}[1]{#1}
\newcommand{\OperatorTok}[1]{\textcolor[rgb]{0.81,0.36,0.00}{\textbf{#1}}}
\newcommand{\OtherTok}[1]{\textcolor[rgb]{0.56,0.35,0.01}{#1}}
\newcommand{\PreprocessorTok}[1]{\textcolor[rgb]{0.56,0.35,0.01}{\textit{#1}}}
\newcommand{\RegionMarkerTok}[1]{#1}
\newcommand{\SpecialCharTok}[1]{\textcolor[rgb]{0.00,0.00,0.00}{#1}}
\newcommand{\SpecialStringTok}[1]{\textcolor[rgb]{0.31,0.60,0.02}{#1}}
\newcommand{\StringTok}[1]{\textcolor[rgb]{0.31,0.60,0.02}{#1}}
\newcommand{\VariableTok}[1]{\textcolor[rgb]{0.00,0.00,0.00}{#1}}
\newcommand{\VerbatimStringTok}[1]{\textcolor[rgb]{0.31,0.60,0.02}{#1}}
\newcommand{\WarningTok}[1]{\textcolor[rgb]{0.56,0.35,0.01}{\textbf{\textit{#1}}}}
\usepackage{longtable,booktabs,array}
\usepackage{calc} % for calculating minipage widths
% Correct order of tables after \paragraph or \subparagraph
\usepackage{etoolbox}
\makeatletter
\patchcmd\longtable{\par}{\if@noskipsec\mbox{}\fi\par}{}{}
\makeatother
% Allow footnotes in longtable head/foot
\IfFileExists{footnotehyper.sty}{\usepackage{footnotehyper}}{\usepackage{footnote}}
\makesavenoteenv{longtable}
\usepackage{graphicx}
\makeatletter
\def\maxwidth{\ifdim\Gin@nat@width>\linewidth\linewidth\else\Gin@nat@width\fi}
\def\maxheight{\ifdim\Gin@nat@height>\textheight\textheight\else\Gin@nat@height\fi}
\makeatother
% Scale images if necessary, so that they will not overflow the page
% margins by default, and it is still possible to overwrite the defaults
% using explicit options in \includegraphics[width, height, ...]{}
\setkeys{Gin}{width=\maxwidth,height=\maxheight,keepaspectratio}
% Set default figure placement to htbp
\makeatletter
\def\fps@figure{htbp}
\makeatother
\setlength{\emergencystretch}{3em} % prevent overfull lines
\providecommand{\tightlist}{%
  \setlength{\itemsep}{0pt}\setlength{\parskip}{0pt}}
\setcounter{secnumdepth}{5}
\usepackage{booktabs}
\usepackage{amsthm}
\makeatletter
\def\thm@space@setup{%
  \thm@preskip=8pt plus 2pt minus 4pt
  \thm@postskip=\thm@preskip
}
\makeatother
\ifLuaTeX
  \usepackage{selnolig}  % disable illegal ligatures
\fi
\usepackage[]{natbib}
\bibliographystyle{apalike}

\title{Kungfu Pandas}
\author{Lê Huỳnh Đức}
\date{2021-05-26}

\begin{document}
\maketitle

{
\setcounter{tocdepth}{1}
\tableofcontents
}
\hypertarget{lux1eddi-nuxf3i-ux111ux1ea7u}{%
\chapter*{Lời nói đầu}\label{lux1eddi-nuxf3i-ux111ux1ea7u}}
\addcontentsline{toc}{chapter}{Lời nói đầu}

\hypertarget{giux1edbi-thiux1ec7u-cuux1ed1n-suxe1ch}{%
\section*{Giới thiệu cuốn sách}\label{giux1edbi-thiux1ec7u-cuux1ed1n-suxe1ch}}
\addcontentsline{toc}{section}{Giới thiệu cuốn sách}

\hypertarget{cuxe0i-ux111ux1eb7t-jupyter-lab}{%
\section*{Cài đặt Jupyter Lab}\label{cuxe0i-ux111ux1eb7t-jupyter-lab}}
\addcontentsline{toc}{section}{Cài đặt Jupyter Lab}

\hypertarget{cuxe0i-ux111ux1eb7t-pandas}{%
\section*{Cài đặt Pandas}\label{cuxe0i-ux111ux1eb7t-pandas}}
\addcontentsline{toc}{section}{Cài đặt Pandas}

\hypertarget{cux1ea5u-truxfac-vuxe0-kiux1ec3u-dux1eef-liux1ec7u}{%
\chapter{Cấu trúc và kiểu dữ liệu}\label{cux1ea5u-truxfac-vuxe0-kiux1ec3u-dux1eef-liux1ec7u}}

\hypertarget{series}{%
\section{Series}\label{series}}

Trong Pandas, \texttt{Series} là mảng 1 chiều bao gồm một danh sách giá trị, và một mảng chứa index
của các giá trị. Trong dữ liệu dảng bảng, mỗi Series được xem như là một cột của bảng đó.
Cách đơn giản để tạo 1 series như sau

\begin{Shaded}
\begin{Highlighting}[]
\NormalTok{s }\OperatorTok{=}\NormalTok{ pd.Series(data, index}\OperatorTok{=}\VariableTok{None}\NormalTok{, name}\OperatorTok{=}\VariableTok{None}\NormalTok{)}
\end{Highlighting}
\end{Shaded}

Trong đó \texttt{data} có thể có dạng:

\begin{itemize}
\item
  numpy.ndarray, List
\item
  Python dict
\item
  Scalar
\end{itemize}

\texttt{index} có thể truyền hoặc không, tùy vào dạng của \texttt{data} mà \texttt{index} sẽ được định nghĩa khác nhau.
\texttt{name} là tên của \texttt{Series}, giá trị này cũng không nhất thiết phải truyền vào.

\hypertarget{cuxe1c-cuxe1ch-khux1edfi-tux1ea1o}{%
\subsection*{Các cách khởi tạo}\label{cuxe1c-cuxe1ch-khux1edfi-tux1ea1o}}
\addcontentsline{toc}{subsection}{Các cách khởi tạo}

\textbf{Khởi tạo Series bằng array}
Khi không truyền giá trị \texttt{index}, \texttt{Series} sẽ mặc định index của nó là 1 mảng số nguyên từ \texttt{0} đến \texttt{len(data)\ -\ 1}

\begin{Shaded}
\begin{Highlighting}[]
\NormalTok{In [}\DecValTok{1}\NormalTok{]: pd.Series(data}\OperatorTok{=}\NormalTok{[}\DecValTok{0}\NormalTok{, }\DecValTok{1}\NormalTok{, }\DecValTok{2}\NormalTok{], index}\OperatorTok{=}\NormalTok{[}\StringTok{"a"}\NormalTok{, }\StringTok{"b"}\NormalTok{, }\StringTok{"c"}\NormalTok{], name}\OperatorTok{=}\StringTok{"meow"}\NormalTok{)}
\NormalTok{Out[}\DecValTok{1}\NormalTok{]:}
\NormalTok{a    }\DecValTok{0}
\NormalTok{b    }\DecValTok{1}
\NormalTok{c    }\DecValTok{2}
\NormalTok{Name: meow, dtype: int64}
\end{Highlighting}
\end{Shaded}

\textbf{Khởi tạo Series bằng dict}

\begin{Shaded}
\begin{Highlighting}[]
\NormalTok{In [}\DecValTok{1}\NormalTok{]: pd.Series(\{}\StringTok{"b"}\NormalTok{:}\DecValTok{1}\NormalTok{, }\StringTok{"a"}\NormalTok{:}\DecValTok{0}\NormalTok{, }\StringTok{"c"}\NormalTok{: }\DecValTok{2}\NormalTok{\})}
\NormalTok{Out[}\DecValTok{1}\NormalTok{]: }
\NormalTok{b    }\DecValTok{1}
\NormalTok{a    }\DecValTok{0}
\NormalTok{c    }\DecValTok{2}
\NormalTok{dtype: int64}
\end{Highlighting}
\end{Shaded}

\begin{rmdnote}
\textbf{\emph{Lưu ý}:}
Trong trường hợp bạn truyền biến \texttt{index} vào, \texttt{Series} sẽ đánh index dựa vào thứ tự trong \texttt{index}, và chỉ chứa các giá trị của dict có key nằm trong \texttt{index}.
Với các giá trị trong biến \texttt{index} không có trong keys của dict, \texttt{Series} sẽ tạo ra các giá trị
bị thiếu \textbf{NaN}.
\end{rmdnote}

\begin{Shaded}
\begin{Highlighting}[]
\NormalTok{In [}\DecValTok{1}\NormalTok{]: pd.Series(\{}\StringTok{"a"}\NormalTok{: }\DecValTok{0}\NormalTok{, }\StringTok{"b"}\NormalTok{: }\DecValTok{1}\NormalTok{, }\StringTok{"c"}\NormalTok{: }\DecValTok{2}\NormalTok{, }\StringTok{"e"}\NormalTok{: }\DecValTok{4}\NormalTok{\}, index}\OperatorTok{=}\NormalTok{[}\StringTok{"b"}\NormalTok{, }\StringTok{"c"}\NormalTok{, }\StringTok{"d"}\NormalTok{, }\StringTok{"a"}\NormalTok{])}
\NormalTok{Out[}\DecValTok{1}\NormalTok{]: }
\NormalTok{b    }\FloatTok{1.0}
\NormalTok{c    }\FloatTok{2.0}
\NormalTok{d    NaN}
\NormalTok{a    }\FloatTok{0.0}
\NormalTok{dtype: float64}
\end{Highlighting}
\end{Shaded}

\begin{rmdnote}
\textbf{\emph{Lưu ý}:} \textbf{NaN} là giá trị mặc định cho dữ liệu bị thiếu trong pandas và giá trị này có kiểu
là \texttt{float64} nên kiểu dữ liệu của \texttt{Series} cũng là \texttt{float64} khác với \texttt{int64} ở ví dụ trước đó.
\end{rmdnote}

\textbf{Khởi tạo Series bằng một giá trị (Scalar)}

\begin{Shaded}
\begin{Highlighting}[]
\NormalTok{In [}\DecValTok{1}\NormalTok{]: pd.Series(data}\OperatorTok{=}\DecValTok{1}\NormalTok{, index}\OperatorTok{=}\NormalTok{[}\StringTok{"a"}\NormalTok{, }\StringTok{"b"}\NormalTok{, }\StringTok{"c"}\NormalTok{])}
\NormalTok{Out[}\DecValTok{1}\NormalTok{]: }
\NormalTok{a    }\DecValTok{1}
\NormalTok{b    }\DecValTok{1}
\NormalTok{c    }\DecValTok{1}
\NormalTok{dtype: int64}
\end{Highlighting}
\end{Shaded}

\hypertarget{mux1ed9t-sux1ed1-thao-tuxe1c-cux1a1-bux1ea3n}{%
\subsection*{Một số thao tác cơ bản}\label{mux1ed9t-sux1ed1-thao-tuxe1c-cux1a1-bux1ea3n}}
\addcontentsline{toc}{subsection}{Một số thao tác cơ bản}

Thao tác trên \texttt{Series} cũng giống với thao tác trên \texttt{numpy.array}. Ngoài ra chúng ta còn có thể
tác với Series dựa vào index

Ví dụ:

\begin{Shaded}
\begin{Highlighting}[]
\NormalTok{In [}\DecValTok{1}\NormalTok{]: s }\OperatorTok{=}\NormalTok{ pd.Series(data}\OperatorTok{=}\NormalTok{[}\DecValTok{0}\NormalTok{, }\DecValTok{1}\NormalTok{, }\DecValTok{2}\NormalTok{, }\DecValTok{3}\NormalTok{, }\DecValTok{4}\NormalTok{, }\DecValTok{5}\NormalTok{], index}\OperatorTok{=}\NormalTok{[}\StringTok{"a"}\NormalTok{, }\StringTok{"b"}\NormalTok{, }\StringTok{"c"}\NormalTok{, }\StringTok{"d"}\NormalTok{, }\StringTok{"e"}\NormalTok{, }\StringTok{"f"}\NormalTok{])}
\end{Highlighting}
\end{Shaded}

\textbf{Lấy theo indice}

\begin{Shaded}
\begin{Highlighting}[]
\NormalTok{In [}\DecValTok{2}\NormalTok{]: s[}\DecValTok{2}\NormalTok{]}
\NormalTok{Out[}\DecValTok{2}\NormalTok{]: }\DecValTok{2}
\end{Highlighting}
\end{Shaded}

\textbf{Lấy theo index}

\begin{Shaded}
\begin{Highlighting}[]
\NormalTok{In [}\DecValTok{3}\NormalTok{]: s[}\StringTok{"c"}\NormalTok{]}
\NormalTok{Out[}\DecValTok{3}\NormalTok{]: }\DecValTok{2} 
\end{Highlighting}
\end{Shaded}

\textbf{Slice indice}

\begin{Shaded}
\begin{Highlighting}[]
\NormalTok{In [}\DecValTok{4}\NormalTok{]: s[}\DecValTok{1}\NormalTok{:}\DecValTok{3}\NormalTok{]}
\NormalTok{Out[}\DecValTok{4}\NormalTok{]:}
\NormalTok{b    }\DecValTok{1}
\NormalTok{d    }\DecValTok{2}
\NormalTok{dtype: int64}
\end{Highlighting}
\end{Shaded}

\textbf{Slice index}

\begin{Shaded}
\begin{Highlighting}[]
\NormalTok{In [}\DecValTok{5}\NormalTok{]: s[}\StringTok{"b"}\NormalTok{:}\StringTok{"c"}\NormalTok{]}
\NormalTok{Out[}\DecValTok{5}\NormalTok{]: }
\NormalTok{b    }\DecValTok{1}
\NormalTok{c    }\DecValTok{2}
\NormalTok{dtype: int64}
\end{Highlighting}
\end{Shaded}

\textbf{List indice}

\begin{Shaded}
\begin{Highlighting}[]
\NormalTok{In [}\DecValTok{6}\NormalTok{]: s[[}\DecValTok{1}\NormalTok{, }\DecValTok{2}\NormalTok{, }\DecValTok{4}\NormalTok{]]}
\NormalTok{Out[}\DecValTok{6}\NormalTok{]:}
\NormalTok{b    }\DecValTok{1}
\NormalTok{c    }\DecValTok{2}
\NormalTok{e    }\DecValTok{4}
\NormalTok{dtype: int64}
\end{Highlighting}
\end{Shaded}

\textbf{List index}

\begin{Shaded}
\begin{Highlighting}[]
\NormalTok{In [}\DecValTok{7}\NormalTok{]: s[[}\StringTok{"b"}\NormalTok{, }\StringTok{"c"}\NormalTok{, }\StringTok{"e"}\NormalTok{]]}
\NormalTok{Out[}\DecValTok{7}\NormalTok{]:}
\NormalTok{b    }\DecValTok{1}
\NormalTok{c    }\DecValTok{2}
\NormalTok{e    }\DecValTok{4}
\NormalTok{dtype: int64}
\end{Highlighting}
\end{Shaded}

\textbf{Điều kiện}

\begin{Shaded}
\begin{Highlighting}[]
\NormalTok{In [}\DecValTok{5}\NormalTok{]: s[s }\OperatorTok{\textgreater{}}\NormalTok{ s.mean()]}
\NormalTok{Out[}\DecValTok{5}\NormalTok{]:}
\NormalTok{d    }\DecValTok{3}
\NormalTok{e    }\DecValTok{4}
\NormalTok{f    }\DecValTok{5}
\NormalTok{dtype: int64}
\end{Highlighting}
\end{Shaded}

\hypertarget{dataframe}{%
\section{DataFrame}\label{dataframe}}

\texttt{DataFrame} là cấu trúc dữ liệu chính và cũng là đặc trưng của pandas. Cũng giống như SQL Table,
\texttt{DataFrame} là một bảng gồm một hay nhiều cột dữ liệu. Hoặc có thể nói rõ hơn là DataFrame là tập
hợp các Series lại với nhau.
Cách khởi tạo DataFrame như sau

\begin{Shaded}
\begin{Highlighting}[]
\NormalTok{df }\OperatorTok{=}\NormalTok{ pd.DataFrame(data}\OperatorTok{=}\VariableTok{None}\NormalTok{, index}\OperatorTok{=}\VariableTok{None}\NormalTok{, columns}\OperatorTok{=}\VariableTok{None}\NormalTok{, dtype}\OperatorTok{=}\VariableTok{None}\NormalTok{, copy}\OperatorTok{=}\VariableTok{False}\NormalTok{)}
\end{Highlighting}
\end{Shaded}

Cũng giống như Series, \texttt{data} của DataFrame có nhiều cách khởi tạo khác nhau như:

\begin{itemize}
\tightlist
\item
  Dict
\item
  Mảng 2 chiều \texttt{numpy.ndarray}, List
\item
  Bản ghi có cấu trúc
\item
  Từ 1 \texttt{Series}
\item
  Từ \texttt{DataFrame} khác
\end{itemize}

Tùy vào cấu trúc của \texttt{data} mà chúng ta có thể bỏ qủua biến \texttt{index}. Biến \texttt{columns} thể hiện tên
của các \texttt{Series}. \texttt{dtype} sẽ định nghĩa các kiểu dữ liêu của dữ liệu, chúng ta sẽ thảo luận về nó
ở phần kế tiếp của chương này. \texttt{copy} dùng để tạo bản sao từ dữ liệu \texttt{data}, nó chỉ ảnh hưởng khi
\texttt{data} là DataFrame khác hoặc numpy.ndarray, việc copy này sẽ tránh trường hợp 2 biến cùng trỏ về
cùng 1 bộ nhớ.

\hypertarget{cuxe1c-cuxe1ch-khux1edfi-tux1ea1o-1}{%
\subsection*{Các cách khởi tạo}\label{cuxe1c-cuxe1ch-khux1edfi-tux1ea1o-1}}
\addcontentsline{toc}{subsection}{Các cách khởi tạo}

\textbf{Khởi tạo DataFrame từ dict của Series}

Khi không truyền biến \texttt{index} vào, thì index của \texttt{DataFrame} sẽ là hợp giữa 2 index của \texttt{Series} và
chúng sẽ được sắp xếp theo thứ tự từ vựng. Nếu ta không truyền \texttt{columns} thì các cột của \texttt{DataFrame} sẽ
được sắp xếp theo thứ tự từ vựng các keys của dict.

Khi truyền biến \texttt{index} vào, tương tự như Series, chỉ những index nằm trong \texttt{index} mới được chọn, còn
những index bị thiếu sẽ được điền giá trị \textbf{NaN}

Khi truyền giá trị \texttt{columns}, DataFrame sẽ chọn những \texttt{Series} thuộc dict có key thuộc \texttt{columns}, giá trị
trong \texttt{columns} không có trong key của dict sẽ được gán \textbf{NaN}

\begin{Shaded}
\begin{Highlighting}[]
\NormalTok{In [}\DecValTok{1}\NormalTok{]: d }\OperatorTok{=}\NormalTok{ \{}
            \StringTok{"one"}\NormalTok{: pd.Series([}\DecValTok{1}\NormalTok{, }\DecValTok{2}\NormalTok{, }\DecValTok{3}\NormalTok{], index}\OperatorTok{=}\NormalTok{[}\StringTok{"c"}\NormalTok{, }\StringTok{"b"}\NormalTok{, }\StringTok{"a"}\NormalTok{]),}
            \StringTok{"two"}\NormalTok{: pd.Series([}\DecValTok{1}\NormalTok{, }\DecValTok{2}\NormalTok{, }\DecValTok{3}\NormalTok{, }\DecValTok{4}\NormalTok{], index}\OperatorTok{=}\NormalTok{[}\StringTok{"c"}\NormalTok{, }\StringTok{"a"}\NormalTok{, }\StringTok{"b"}\NormalTok{, }\StringTok{"d"}\NormalTok{])}
\NormalTok{            \}}
\NormalTok{In [}\DecValTok{2}\NormalTok{]: pd.DataFrame(d)}
\NormalTok{Out[}\DecValTok{2}\NormalTok{]:}
\NormalTok{   one  two}
\NormalTok{a  }\FloatTok{3.0}    \DecValTok{2}
\NormalTok{b  }\FloatTok{2.0}    \DecValTok{3}
\NormalTok{c  }\FloatTok{1.0}    \DecValTok{1}
\NormalTok{d  NaN    }\DecValTok{4}

\NormalTok{In [}\DecValTok{3}\NormalTok{]: pd.DataFrame(d, index}\OperatorTok{=}\NormalTok{[}\StringTok{"d"}\NormalTok{, }\StringTok{"b"}\NormalTok{, }\StringTok{"a"}\NormalTok{])}
\NormalTok{Out[}\DecValTok{3}\NormalTok{]: }
\NormalTok{   one  two}
\NormalTok{d  NaN    }\DecValTok{4}
\NormalTok{b  }\FloatTok{2.0}    \DecValTok{3}
\NormalTok{a  }\FloatTok{3.0}    \DecValTok{2}

\NormalTok{In [}\DecValTok{4}\NormalTok{]: pd.DataFrame(d, index}\OperatorTok{=}\NormalTok{[}\StringTok{"d"}\NormalTok{, }\StringTok{"b"}\NormalTok{, }\StringTok{"a"}\NormalTok{], columns}\OperatorTok{=}\NormalTok{[}\StringTok{"two"}\NormalTok{, }\StringTok{"three"}\NormalTok{])}
\NormalTok{Out[}\DecValTok{4}\NormalTok{]:}
\NormalTok{   two  three}
\NormalTok{d    }\DecValTok{4}\NormalTok{    NaN}
\NormalTok{b    }\DecValTok{3}\NormalTok{    NaN}
\NormalTok{a    }\DecValTok{2}\NormalTok{    NaN}
\end{Highlighting}
\end{Shaded}

\textbf{Khởi tạo DataFrame từ Mảng 2 chiều}

\textbf{Khởi tạo DataFrame từ danh sách các dict}

\textbf{Khởi tạo DataFrame từ bảng có cấu trúc}

\hypertarget{data-type-trong-pandas}{%
\section{Data type trong pandas}\label{data-type-trong-pandas}}

\begin{longtable}[]{@{}
  >{\raggedright\arraybackslash}p{(\columnwidth - 4\tabcolsep) * \real{0.33}}
  >{\raggedright\arraybackslash}p{(\columnwidth - 4\tabcolsep) * \real{0.25}}
  >{\raggedright\arraybackslash}p{(\columnwidth - 4\tabcolsep) * \real{0.21}}@{}}
\toprule
Các kiểu dữ liệu
phổ biến & Numpy/Pandas
object & Hiển thị \\
\midrule
\endhead
Boolean & np.bool & \emph{bool} \\
Integer & np.int & \emph{int} \\
Float & np.float & \emph{float} \\
Object & np.object & \emph{O, object} \\
Datetime & np.datetime64,
pd.Timestamp & \emph{datetime64} \\
Timedelta & np.timedelta64,
pd.Timedelta & \emph{timedelta64} \\
Category & pd.categorical & \emph{category} \\
\bottomrule
\end{longtable}

\hypertarget{nhux1eadp-xuux1ea5t-trong-pandas}{%
\chapter{Nhập xuất trong pandas}\label{nhux1eadp-xuux1ea5t-trong-pandas}}

\hypertarget{ux111ux1ecdc-vuxe0-lux1b0u-file}{%
\section{Đọc và lưu file}\label{ux111ux1ecdc-vuxe0-lux1b0u-file}}

\hypertarget{cux1ea5u-huxecnh-pandas}{%
\section{Cấu hình pandas}\label{cux1ea5u-huxecnh-pandas}}

\hypertarget{mux1ed9t-sux1ed1-huxe0m-cux1a1-bux1ea3n}{%
\chapter{Một số hàm cơ bản}\label{mux1ed9t-sux1ed1-huxe0m-cux1a1-bux1ea3n}}

\hypertarget{lux1eb7p-trong-pandas}{%
\chapter{Lặp trong Pandas}\label{lux1eb7p-trong-pandas}}

\hypertarget{sux1eed-dux1ee5ng-vectorizer}{%
\section{Sử dụng vectorizer}\label{sux1eed-dux1ee5ng-vectorizer}}

\hypertarget{sux1eed-dux1ee5ng-apply}{%
\section{Sử dụng apply}\label{sux1eed-dux1ee5ng-apply}}

\hypertarget{sux1eed-dux1ee5ng-iterator}{%
\section{Sử dụng iterator}\label{sux1eed-dux1ee5ng-iterator}}

\hypertarget{xux1eed-luxfd-song-song-trong-pandas}{%
\section{Xử lý song song trong pandas}\label{xux1eed-luxfd-song-song-trong-pandas}}

\hypertarget{select-vuxe0-filter}{%
\chapter{Select và Filter}\label{select-vuxe0-filter}}

\hypertarget{index}{%
\section{Index}\label{index}}

\hypertarget{loc-vuxe0-iloc}{%
\section{loc và iloc}\label{loc-vuxe0-iloc}}

\hypertarget{lux1ecdc-theo-ux111iux1ec1u-kiux1ec7n}{%
\section{Lọc theo điều kiện}\label{lux1ecdc-theo-ux111iux1ec1u-kiux1ec7n}}

\hypertarget{cuxe1c-cuxe1ch-phux1ed1i-hux1ee3p-nhiux1ec1u-bux1ea3ng-vux1edbi-nhau}{%
\chapter{Các cách phối hợp nhiều bảng với nhau}\label{cuxe1c-cuxe1ch-phux1ed1i-hux1ee3p-nhiux1ec1u-bux1ea3ng-vux1edbi-nhau}}

\hypertarget{join}{%
\section{Join}\label{join}}

\hypertarget{merge}{%
\section{Merge}\label{merge}}

\hypertarget{concat}{%
\section{Concat}\label{concat}}

\hypertarget{groupby-vuxe0-aggregate}{%
\chapter{Groupby và Aggregate}\label{groupby-vuxe0-aggregate}}

\hypertarget{luxe0m-viux1ec7c-vux1edbi-1-sux1ed1-kiux1ec3u-dux1eef-liux1ec7u}{%
\chapter{Làm việc với 1 số kiểu dữ liệu}\label{luxe0m-viux1ec7c-vux1edbi-1-sux1ed1-kiux1ec3u-dux1eef-liux1ec7u}}

\hypertarget{xux1eed-luxfd-dux1eef-liux1ec7u-dux1ea1ng-text}{%
\section{Xử lý dữ liệu dạng text}\label{xux1eed-luxfd-dux1eef-liux1ec7u-dux1ea1ng-text}}

\hypertarget{xux1eed-luxfd-dux1eef-liux1ec7u-dux1ea1ng-timestamp}{%
\section{Xử lý dữ liệu dạng timestamp}\label{xux1eed-luxfd-dux1eef-liux1ec7u-dux1ea1ng-timestamp}}

\hypertarget{category-trong-pandas}{%
\section{Category trong pandas}\label{category-trong-pandas}}

\hypertarget{xux1eed-luxfd-missing-data}{%
\section{Xử lý Missing data}\label{xux1eed-luxfd-missing-data}}

\hypertarget{mux1ed9t-sux1ed1-kiux1ebfn-thux1ee9c-nuxe2ng-cao}{%
\chapter{Một số kiến thức nâng cao}\label{mux1ed9t-sux1ed1-kiux1ebfn-thux1ee9c-nuxe2ng-cao}}

\hypertarget{multiindex}{%
\section{MultiIndex}\label{multiindex}}

\hypertarget{pivot-vuxe0-merge}{%
\section{Pivot và Merge}\label{pivot-vuxe0-merge}}

\hypertarget{resample}{%
\section{Resample}\label{resample}}

\hypertarget{window}{%
\section{Window}\label{window}}

\hypertarget{anomaly-detection-project}{%
\chapter{Anomaly Detection Project}\label{anomaly-detection-project}}

\hypertarget{visualize-vux1edbi-matplotlib}{%
\chapter{Visualize với Matplotlib}\label{visualize-vux1edbi-matplotlib}}

  \bibliography{book.bib,packages.bib}

\end{document}
