% Options for packages loaded elsewhere
\PassOptionsToPackage{unicode}{hyperref}
\PassOptionsToPackage{hyphens}{url}
%
\documentclass[
]{book}
\usepackage{amsmath,amssymb}
\usepackage{lmodern}
\usepackage{iftex}
\ifPDFTeX
  \usepackage[T1]{fontenc}
  \usepackage[utf8]{inputenc}
  \usepackage{textcomp} % provide euro and other symbols
\else % if luatex or xetex
  \usepackage{unicode-math}
  \defaultfontfeatures{Scale=MatchLowercase}
  \defaultfontfeatures[\rmfamily]{Ligatures=TeX,Scale=1}
\fi
% Use upquote if available, for straight quotes in verbatim environments
\IfFileExists{upquote.sty}{\usepackage{upquote}}{}
\IfFileExists{microtype.sty}{% use microtype if available
  \usepackage[]{microtype}
  \UseMicrotypeSet[protrusion]{basicmath} % disable protrusion for tt fonts
}{}
\makeatletter
\@ifundefined{KOMAClassName}{% if non-KOMA class
  \IfFileExists{parskip.sty}{%
    \usepackage{parskip}
  }{% else
    \setlength{\parindent}{0pt}
    \setlength{\parskip}{6pt plus 2pt minus 1pt}}
}{% if KOMA class
  \KOMAoptions{parskip=half}}
\makeatother
\usepackage{xcolor}
\IfFileExists{xurl.sty}{\usepackage{xurl}}{} % add URL line breaks if available
\IfFileExists{bookmark.sty}{\usepackage{bookmark}}{\usepackage{hyperref}}
\hypersetup{
  pdftitle={Kungfu Pandas},
  pdfauthor={Lê Huỳnh Đức},
  hidelinks,
  pdfcreator={LaTeX via pandoc}}
\urlstyle{same} % disable monospaced font for URLs
\usepackage{color}
\usepackage{fancyvrb}
\newcommand{\VerbBar}{|}
\newcommand{\VERB}{\Verb[commandchars=\\\{\}]}
\DefineVerbatimEnvironment{Highlighting}{Verbatim}{commandchars=\\\{\}}
% Add ',fontsize=\small' for more characters per line
\usepackage{framed}
\definecolor{shadecolor}{RGB}{248,248,248}
\newenvironment{Shaded}{\begin{snugshade}}{\end{snugshade}}
\newcommand{\AlertTok}[1]{\textcolor[rgb]{0.94,0.16,0.16}{#1}}
\newcommand{\AnnotationTok}[1]{\textcolor[rgb]{0.56,0.35,0.01}{\textbf{\textit{#1}}}}
\newcommand{\AttributeTok}[1]{\textcolor[rgb]{0.77,0.63,0.00}{#1}}
\newcommand{\BaseNTok}[1]{\textcolor[rgb]{0.00,0.00,0.81}{#1}}
\newcommand{\BuiltInTok}[1]{#1}
\newcommand{\CharTok}[1]{\textcolor[rgb]{0.31,0.60,0.02}{#1}}
\newcommand{\CommentTok}[1]{\textcolor[rgb]{0.56,0.35,0.01}{\textit{#1}}}
\newcommand{\CommentVarTok}[1]{\textcolor[rgb]{0.56,0.35,0.01}{\textbf{\textit{#1}}}}
\newcommand{\ConstantTok}[1]{\textcolor[rgb]{0.00,0.00,0.00}{#1}}
\newcommand{\ControlFlowTok}[1]{\textcolor[rgb]{0.13,0.29,0.53}{\textbf{#1}}}
\newcommand{\DataTypeTok}[1]{\textcolor[rgb]{0.13,0.29,0.53}{#1}}
\newcommand{\DecValTok}[1]{\textcolor[rgb]{0.00,0.00,0.81}{#1}}
\newcommand{\DocumentationTok}[1]{\textcolor[rgb]{0.56,0.35,0.01}{\textbf{\textit{#1}}}}
\newcommand{\ErrorTok}[1]{\textcolor[rgb]{0.64,0.00,0.00}{\textbf{#1}}}
\newcommand{\ExtensionTok}[1]{#1}
\newcommand{\FloatTok}[1]{\textcolor[rgb]{0.00,0.00,0.81}{#1}}
\newcommand{\FunctionTok}[1]{\textcolor[rgb]{0.00,0.00,0.00}{#1}}
\newcommand{\ImportTok}[1]{#1}
\newcommand{\InformationTok}[1]{\textcolor[rgb]{0.56,0.35,0.01}{\textbf{\textit{#1}}}}
\newcommand{\KeywordTok}[1]{\textcolor[rgb]{0.13,0.29,0.53}{\textbf{#1}}}
\newcommand{\NormalTok}[1]{#1}
\newcommand{\OperatorTok}[1]{\textcolor[rgb]{0.81,0.36,0.00}{\textbf{#1}}}
\newcommand{\OtherTok}[1]{\textcolor[rgb]{0.56,0.35,0.01}{#1}}
\newcommand{\PreprocessorTok}[1]{\textcolor[rgb]{0.56,0.35,0.01}{\textit{#1}}}
\newcommand{\RegionMarkerTok}[1]{#1}
\newcommand{\SpecialCharTok}[1]{\textcolor[rgb]{0.00,0.00,0.00}{#1}}
\newcommand{\SpecialStringTok}[1]{\textcolor[rgb]{0.31,0.60,0.02}{#1}}
\newcommand{\StringTok}[1]{\textcolor[rgb]{0.31,0.60,0.02}{#1}}
\newcommand{\VariableTok}[1]{\textcolor[rgb]{0.00,0.00,0.00}{#1}}
\newcommand{\VerbatimStringTok}[1]{\textcolor[rgb]{0.31,0.60,0.02}{#1}}
\newcommand{\WarningTok}[1]{\textcolor[rgb]{0.56,0.35,0.01}{\textbf{\textit{#1}}}}
\usepackage{longtable,booktabs,array}
\usepackage{calc} % for calculating minipage widths
% Correct order of tables after \paragraph or \subparagraph
\usepackage{etoolbox}
\makeatletter
\patchcmd\longtable{\par}{\if@noskipsec\mbox{}\fi\par}{}{}
\makeatother
% Allow footnotes in longtable head/foot
\IfFileExists{footnotehyper.sty}{\usepackage{footnotehyper}}{\usepackage{footnote}}
\makesavenoteenv{longtable}
\setlength{\emergencystretch}{3em} % prevent overfull lines
\providecommand{\tightlist}{%
  \setlength{\itemsep}{0pt}\setlength{\parskip}{0pt}}
\setcounter{secnumdepth}{5}
\usepackage{booktabs}
\usepackage{longtable}
\usepackage[bf,singlelinecheck=off]{caption}
\usepackage{graphicx}
\usepackage{Alegreya}
\usepackage[scale=.7]{sourcecodepro}

\usepackage{framed,color}
\definecolor{shadecolor}{RGB}{248,248,248}

\renewcommand{\textfraction}{0.05}
\renewcommand{\topfraction}{0.8}
\renewcommand{\bottomfraction}{0.8}
\renewcommand{\floatpagefraction}{0.75}

\renewenvironment{quote}{\begin{VF}}{\end{VF}}
\let\oldhref\href
\renewcommand{\href}[2]{#2\footnote{\url{#1}}}

\ifxetex
  \usepackage{letltxmacro}
  \setlength{\XeTeXLinkMargin}{1pt}
  \LetLtxMacro\SavedIncludeGraphics\includegraphics
  \def\includegraphics#1#{% #1 catches optional stuff (star/opt. arg.)
    \IncludeGraphicsAux{#1}%
  }%
  \newcommand*{\IncludeGraphicsAux}[2]{%
    \XeTeXLinkBox{%
      \SavedIncludeGraphics#1{#2}%
    }%
  }%
\fi

\makeatletter
\newenvironment{kframe}{%
\medskip{}
\setlength{\fboxsep}{.8em}
 \def\at@end@of@kframe{}%
 \ifinner\ifhmode%
  \def\at@end@of@kframe{\end{minipage}}%
  \begin{minipage}{\columnwidth}%
 \fi\fi%
 \def\FrameCommand##1{\hskip\@totalleftmargin \hskip-\fboxsep
 \colorbox{shadecolor}{##1}\hskip-\fboxsep
     % There is no \\@totalrightmargin, so:
     \hskip-\linewidth \hskip-\@totalleftmargin \hskip\columnwidth}%
 \MakeFramed {\advance\hsize-\width
   \@totalleftmargin\z@ \linewidth\hsize
   \@setminipage}}%
 {\par\unskip\endMakeFramed%
 \at@end@of@kframe}
\makeatother

\makeatletter
\@ifundefined{Shaded}{
}{\renewenvironment{Shaded}{\begin{kframe}}{\end{kframe}}}
\makeatother

\newenvironment{rmdblock}[1]
  {
  \begin{itemize}
  \renewcommand{\labelitemi}{
    \raisebox{-.7\height}[0pt][0pt]{
      {\setkeys{Gin}{width=3em,keepaspectratio}\includegraphics{images/#1}}
    }
  }
  \setlength{\fboxsep}{1em}
  \begin{kframe}
  \item
  }
  {
  \end{kframe}
  \end{itemize}
  }
\newenvironment{rmdnote}
  {\begin{rmdblock}{note}}
  {\end{rmdblock}}
\newenvironment{rmdcaution}
  {\begin{rmdblock}{caution}}
  {\end{rmdblock}}
\newenvironment{rmdimportant}
  {\begin{rmdblock}{important}}
  {\end{rmdblock}}
\newenvironment{rmdtip}
  {\begin{rmdblock}{tip}}
  {\end{rmdblock}}
\newenvironment{rmdwarning}
  {\begin{rmdblock}{warning}}
  {\end{rmdblock}}

\usepackage{makeidx}
\makeindex

\urlstyle{tt}

\usepackage{amsthm}
\makeatletter
\def\thm@space@setup{%
  \thm@preskip=8pt plus 2pt minus 4pt
  \thm@postskip=\thm@preskip
}
\makeatother

\frontmatter
\ifLuaTeX
  \usepackage{selnolig}  % disable illegal ligatures
\fi
\usepackage[]{natbib}
\bibliographystyle{plainnat}

\title{Kungfu Pandas}
\author{Lê Huỳnh Đức}
\date{2021-06-03}

\begin{document}
\maketitle

%\cleardoublepage\newpage\thispagestyle{empty}\null
%\cleardoublepage\newpage\thispagestyle{empty}\null
%\cleardoublepage\newpage
\thispagestyle{empty}
\begin{center}
\includegraphics{images/dedication.pdf}
\end{center}

\setlength{\abovedisplayskip}{-5pt}
\setlength{\abovedisplayshortskip}{-5pt}

{
\setcounter{tocdepth}{2}
\tableofcontents
}
\hypertarget{lux1eddi-nuxf3i-ux111ux1ea7u}{%
\chapter*{Lời nói đầu}\label{lux1eddi-nuxf3i-ux111ux1ea7u}}


\hypertarget{giux1edbi-thiux1ec7u-cuux1ed1n-suxe1ch}{%
\section*{Giới thiệu cuốn sách}\label{giux1edbi-thiux1ec7u-cuux1ed1n-suxe1ch}}


\hypertarget{cuxe0i-ux111ux1eb7t-jupyter-lab}{%
\section*{Cài đặt Jupyter Lab}\label{cuxe0i-ux111ux1eb7t-jupyter-lab}}


\hypertarget{cuxe0i-ux111ux1eb7t-pandas}{%
\section*{Cài đặt Pandas}\label{cuxe0i-ux111ux1eb7t-pandas}}


\hypertarget{cux1ea5u-truxfac-vuxe0-kiux1ec3u-dux1eef-liux1ec7u}{%
\chapter{Cấu trúc và kiểu dữ liệu}\label{cux1ea5u-truxfac-vuxe0-kiux1ec3u-dux1eef-liux1ec7u}}

\hypertarget{series}{%
\section{Series}\label{series}}

Trong Pandas, \texttt{Series} là mảng 1 chiều bao gồm một danh sách giá trị, và một mảng chứa index
của các giá trị. Trong dữ liệu dảng bảng, mỗi Series được xem như là một cột của bảng đó.
Cách đơn giản để tạo 1 series như sau

\begin{Shaded}
\begin{Highlighting}[]
\NormalTok{s }\OperatorTok{=}\NormalTok{ pd.Series(data, index}\OperatorTok{=}\VariableTok{None}\NormalTok{, name}\OperatorTok{=}\VariableTok{None}\NormalTok{)}
\end{Highlighting}
\end{Shaded}

Trong đó \texttt{data} có thể có dạng:

\begin{itemize}
\item
  \texttt{numpy.ndarray}, \texttt{List}
\item
  Python \texttt{dict}
\item
  \texttt{Scalar}
\end{itemize}

\texttt{index} có thể truyền hoặc không, tùy vào dạng của \texttt{data} mà \texttt{index} sẽ được định nghĩa khác nhau.
\texttt{name} là tên của \texttt{Series}, giá trị này cũng không nhất thiết phải truyền vào.

\hypertarget{cuxe1c-cuxe1ch-khux1edfi-tux1ea1o}{%
\subsection{Các cách khởi tạo}\label{cuxe1c-cuxe1ch-khux1edfi-tux1ea1o}}

\textbf{Khởi tạo Series bằng array}

Khi không truyền giá trị \texttt{index}, \texttt{Series} sẽ mặc định index của nó là 1 mảng số nguyên từ \texttt{0} đến \texttt{len(data)\ -\ 1}

\begin{Shaded}
\begin{Highlighting}[]
\NormalTok{In [}\DecValTok{1}\NormalTok{]: pd.Series(data}\OperatorTok{=}\NormalTok{[}\DecValTok{0}\NormalTok{, }\DecValTok{1}\NormalTok{, }\DecValTok{2}\NormalTok{], index}\OperatorTok{=}\NormalTok{[}\StringTok{"a"}\NormalTok{, }\StringTok{"b"}\NormalTok{, }\StringTok{"c"}\NormalTok{], name}\OperatorTok{=}\StringTok{"meow"}\NormalTok{)}
\NormalTok{Out[}\DecValTok{1}\NormalTok{]:}
\NormalTok{a    }\DecValTok{0}
\NormalTok{b    }\DecValTok{1}
\NormalTok{c    }\DecValTok{2}
\NormalTok{Name: meow, dtype: int64}
\end{Highlighting}
\end{Shaded}

\textbf{Khởi tạo Series bằng dict}

\begin{Shaded}
\begin{Highlighting}[]
\NormalTok{In [}\DecValTok{1}\NormalTok{]: pd.Series(\{}\StringTok{"b"}\NormalTok{: }\DecValTok{1}\NormalTok{, }\StringTok{"a"}\NormalTok{:}\DecValTok{0}\NormalTok{, }\StringTok{"c"}\NormalTok{: }\DecValTok{2}\NormalTok{\})}
\NormalTok{Out[}\DecValTok{1}\NormalTok{]: }
\NormalTok{b    }\DecValTok{1}
\NormalTok{a    }\DecValTok{0}
\NormalTok{c    }\DecValTok{2}
\NormalTok{dtype: int64}
\end{Highlighting}
\end{Shaded}

\begin{rmdnote}
\textbf{\emph{Lưu ý}:}
Trong trường hợp bạn truyền biến \texttt{index} vào, \texttt{Series} sẽ đánh index dựa vào thứ tự trong \texttt{index}, và chỉ chứa các giá trị của dict có key nằm trong \texttt{index}.
Với các giá trị trong biến \texttt{index} không có trong keys của dict, \texttt{Series} sẽ tạo ra các giá trị bị thiếu \texttt{NaN}.
\end{rmdnote}

\begin{Shaded}
\begin{Highlighting}[]
\NormalTok{In [}\DecValTok{1}\NormalTok{]: pd.Series(\{}\StringTok{"a"}\NormalTok{: }\DecValTok{0}\NormalTok{, }\StringTok{"b"}\NormalTok{: }\DecValTok{1}\NormalTok{, }\StringTok{"c"}\NormalTok{: }\DecValTok{2}\NormalTok{, }\StringTok{"e"}\NormalTok{: }\DecValTok{4}\NormalTok{\}, index}\OperatorTok{=}\NormalTok{[}\StringTok{"b"}\NormalTok{, }\StringTok{"c"}\NormalTok{, }\StringTok{"d"}\NormalTok{, }\StringTok{"a"}\NormalTok{])}
\NormalTok{Out[}\DecValTok{1}\NormalTok{]: }
\NormalTok{b    }\FloatTok{1.0}
\NormalTok{c    }\FloatTok{2.0}
\NormalTok{d    NaN}
\NormalTok{a    }\FloatTok{0.0}
\NormalTok{dtype: float64}
\end{Highlighting}
\end{Shaded}

\begin{rmdnote}
\textbf{\emph{Lưu ý}:}
\texttt{NaN} là giá trị mặc định cho dữ liệu bị thiếu trong pandas và giá trị này có kiểu
là \texttt{float64} nên kiểu dữ liệu của \texttt{Series} cũng là \texttt{float64} khác với \texttt{int64} ở ví dụ trước đó.
\end{rmdnote}

\textbf{Khởi tạo Series bằng một giá trị (Scalar)}

\begin{Shaded}
\begin{Highlighting}[]
\NormalTok{In [}\DecValTok{1}\NormalTok{]: pd.Series(data}\OperatorTok{=}\DecValTok{1}\NormalTok{, index}\OperatorTok{=}\NormalTok{[}\StringTok{"a"}\NormalTok{, }\StringTok{"b"}\NormalTok{, }\StringTok{"c"}\NormalTok{])}
\NormalTok{Out[}\DecValTok{1}\NormalTok{]: }
\NormalTok{a    }\DecValTok{1}
\NormalTok{b    }\DecValTok{1}
\NormalTok{c    }\DecValTok{1}
\NormalTok{dtype: int64}
\end{Highlighting}
\end{Shaded}

\hypertarget{mux1ed9t-sux1ed1-thao-tuxe1c-cux1a1-bux1ea3n}{%
\subsection{Một số thao tác cơ bản}\label{mux1ed9t-sux1ed1-thao-tuxe1c-cux1a1-bux1ea3n}}

Thao tác trên \texttt{Series} cũng giống với thao tác trên \texttt{numpy.array}. Ngoài ra chúng ta còn có thể
tác với Series dựa vào index

Ví dụ:

\begin{Shaded}
\begin{Highlighting}[]
\NormalTok{In [}\DecValTok{1}\NormalTok{]: s }\OperatorTok{=}\NormalTok{ pd.Series(data}\OperatorTok{=}\NormalTok{[}\DecValTok{0}\NormalTok{, }\DecValTok{1}\NormalTok{, }\DecValTok{2}\NormalTok{, }\DecValTok{3}\NormalTok{, }\DecValTok{4}\NormalTok{, }\DecValTok{5}\NormalTok{], index}\OperatorTok{=}\NormalTok{[}\StringTok{"a"}\NormalTok{, }\StringTok{"b"}\NormalTok{, }\StringTok{"c"}\NormalTok{, }\StringTok{"d"}\NormalTok{, }\StringTok{"e"}\NormalTok{, }\StringTok{"f"}\NormalTok{])}
\end{Highlighting}
\end{Shaded}

\textbf{Hiển thị toàn bộ giá trị của Series}
Ta gọi thuộc tính \texttt{.values}

\begin{Shaded}
\begin{Highlighting}[]
\NormalTok{In [}\DecValTok{1}\NormalTok{]: s.values}
\NormalTok{Out[}\DecValTok{1}\NormalTok{]:}
\NormalTok{array([}\DecValTok{0}\NormalTok{, }\DecValTok{1}\NormalTok{, }\DecValTok{2}\NormalTok{, }\DecValTok{3}\NormalTok{, }\DecValTok{4}\NormalTok{, }\DecValTok{5}\NormalTok{])}
\end{Highlighting}
\end{Shaded}

\textbf{Lấy theo indice}

\begin{Shaded}
\begin{Highlighting}[]
\NormalTok{In [}\DecValTok{2}\NormalTok{]: s[}\DecValTok{2}\NormalTok{]}
\NormalTok{Out[}\DecValTok{2}\NormalTok{]: }\DecValTok{2}
\end{Highlighting}
\end{Shaded}

\textbf{Lấy theo index}

\begin{Shaded}
\begin{Highlighting}[]
\NormalTok{In [}\DecValTok{3}\NormalTok{]: s[}\StringTok{"c"}\NormalTok{]}
\NormalTok{Out[}\DecValTok{3}\NormalTok{]: }\DecValTok{2} 
\end{Highlighting}
\end{Shaded}

\textbf{Slice indice}

\begin{Shaded}
\begin{Highlighting}[]
\NormalTok{In [}\DecValTok{4}\NormalTok{]: s[}\DecValTok{1}\NormalTok{:}\DecValTok{3}\NormalTok{]}
\NormalTok{Out[}\DecValTok{4}\NormalTok{]:}
\NormalTok{b    }\DecValTok{1}
\NormalTok{d    }\DecValTok{2}
\NormalTok{dtype: int64}
\end{Highlighting}
\end{Shaded}

\textbf{Slice index}

\begin{Shaded}
\begin{Highlighting}[]
\NormalTok{In [}\DecValTok{5}\NormalTok{]: s[}\StringTok{"b"}\NormalTok{:}\StringTok{"c"}\NormalTok{]}
\NormalTok{Out[}\DecValTok{5}\NormalTok{]: }
\NormalTok{b    }\DecValTok{1}
\NormalTok{c    }\DecValTok{2}
\NormalTok{dtype: int64}
\end{Highlighting}
\end{Shaded}

\textbf{List indice}

\begin{Shaded}
\begin{Highlighting}[]
\NormalTok{In [}\DecValTok{6}\NormalTok{]: s[[}\DecValTok{1}\NormalTok{, }\DecValTok{2}\NormalTok{, }\DecValTok{4}\NormalTok{]]}
\NormalTok{Out[}\DecValTok{6}\NormalTok{]:}
\NormalTok{b    }\DecValTok{1}
\NormalTok{c    }\DecValTok{2}
\NormalTok{e    }\DecValTok{4}
\NormalTok{dtype: int64}
\end{Highlighting}
\end{Shaded}

\textbf{List index}

\begin{Shaded}
\begin{Highlighting}[]
\NormalTok{In [}\DecValTok{7}\NormalTok{]: s[[}\StringTok{"b"}\NormalTok{, }\StringTok{"c"}\NormalTok{, }\StringTok{"e"}\NormalTok{]]}
\NormalTok{Out[}\DecValTok{7}\NormalTok{]:}
\NormalTok{b    }\DecValTok{1}
\NormalTok{c    }\DecValTok{2}
\NormalTok{e    }\DecValTok{4}
\NormalTok{dtype: int64}
\end{Highlighting}
\end{Shaded}

\textbf{Điều kiện}

\begin{Shaded}
\begin{Highlighting}[]
\NormalTok{In [}\DecValTok{5}\NormalTok{]: s[s }\OperatorTok{\textgreater{}}\NormalTok{ s.mean()]}
\NormalTok{Out[}\DecValTok{5}\NormalTok{]:}
\NormalTok{d    }\DecValTok{3}
\NormalTok{e    }\DecValTok{4}
\NormalTok{f    }\DecValTok{5}
\NormalTok{dtype: int64}
\end{Highlighting}
\end{Shaded}

\hypertarget{dataframe}{%
\section{DataFrame}\label{dataframe}}

\texttt{DataFrame} là cấu trúc dữ liệu chính và cũng là đặc trưng của pandas. Cũng giống như SQL Table,
\texttt{DataFrame} là một bảng gồm một hay nhiều cột dữ liệu. Hoặc có thể nói rõ hơn là DataFrame là tập
hợp các Series lại với nhau.

Cách khởi tạo DataFrame như sau

\begin{Shaded}
\begin{Highlighting}[]
\NormalTok{df }\OperatorTok{=}\NormalTok{ pd.DataFrame(data}\OperatorTok{=}\VariableTok{None}\NormalTok{, index}\OperatorTok{=}\VariableTok{None}\NormalTok{, columns}\OperatorTok{=}\VariableTok{None}\NormalTok{, dtype}\OperatorTok{=}\VariableTok{None}\NormalTok{, copy}\OperatorTok{=}\VariableTok{False}\NormalTok{)}
\end{Highlighting}
\end{Shaded}

Cũng giống như Series, \texttt{data} của DataFrame có nhiều cách khởi tạo khác nhau như:

\begin{itemize}
\tightlist
\item
  \texttt{dict} của Series, \texttt{dict} của \texttt{numpy.array}/\texttt{List}
\item
  Mảng 2 chiều \texttt{numpy.ndarray}, \texttt{List} của \texttt{List}
\item
  \href{https://numpy.org/doc/stable/user/basics.rec.html}{Mảng có cấu trúc}
\item
  Từ 1 \texttt{Series}
\item
  Từ \texttt{DataFrame} khác
\end{itemize}

Tùy vào cấu trúc của \texttt{data} mà chúng ta có thể bỏ qua biến \texttt{index}. Biến \texttt{columns} thể hiện tên
của các \texttt{Series}. \texttt{dtype} sẽ định nghĩa các kiểu dữ liêu của dữ liệu, chúng ta sẽ thảo luận về nó
ở phần kế tiếp của chương này. \texttt{copy} dùng để tạo bản sao từ dữ liệu \texttt{data}, nó chỉ ảnh hưởng khi
\texttt{data} là DataFrame khác hoặc numpy.ndarray, việc copy này sẽ tránh trường hợp 2 biến cùng trỏ về
cùng 1 bộ nhớ.

\hypertarget{cuxe1c-cuxe1ch-khux1edfi-tux1ea1o-1}{%
\subsection{Các cách khởi tạo}\label{cuxe1c-cuxe1ch-khux1edfi-tux1ea1o-1}}

\textbf{Khởi tạo DataFrame từ dict của Series}

Khi không truyền biến \texttt{index} vào, thì index của \texttt{DataFrame} sẽ là hợp giữa 2 index của \texttt{Series} và
chúng sẽ được sắp xếp theo thứ tự từ vựng. Nếu ta không truyền \texttt{columns} thì các cột của \texttt{DataFrame} sẽ
được sắp xếp theo thứ tự truyền vào các keys của dict.

Khi truyền biến \texttt{index} vào, tương tự như Series, chỉ những index nằm trong \texttt{index} mới được chọn, còn
những index bị thiếu sẽ được điền giá trị \texttt{NaN}

Khi truyền giá trị \texttt{columns}, DataFrame sẽ chọn những \texttt{Series} thuộc dict có key thuộc \texttt{columns}, giá trị
trong \texttt{columns} không có trong key của dict sẽ được gán \texttt{NaN}

\begin{Shaded}
\begin{Highlighting}[]
\NormalTok{In [}\DecValTok{1}\NormalTok{]: d }\OperatorTok{=}\NormalTok{ \{}
            \StringTok{"one"}\NormalTok{: pd.Series([}\DecValTok{1}\NormalTok{, }\DecValTok{2}\NormalTok{, }\DecValTok{3}\NormalTok{], index}\OperatorTok{=}\NormalTok{[}\StringTok{"c"}\NormalTok{, }\StringTok{"b"}\NormalTok{, }\StringTok{"a"}\NormalTok{]),}
            \StringTok{"two"}\NormalTok{: pd.Series([}\DecValTok{1}\NormalTok{, }\DecValTok{2}\NormalTok{, }\DecValTok{3}\NormalTok{, }\DecValTok{4}\NormalTok{], index}\OperatorTok{=}\NormalTok{[}\StringTok{"c"}\NormalTok{, }\StringTok{"a"}\NormalTok{, }\StringTok{"b"}\NormalTok{, }\StringTok{"d"}\NormalTok{])}
\NormalTok{        \}}
\NormalTok{In [}\DecValTok{2}\NormalTok{]: pd.DataFrame(d)}
\NormalTok{Out[}\DecValTok{2}\NormalTok{]:}
\NormalTok{   one  two}
\NormalTok{a  }\FloatTok{3.0}    \DecValTok{2}
\NormalTok{b  }\FloatTok{2.0}    \DecValTok{3}
\NormalTok{c  }\FloatTok{1.0}    \DecValTok{1}
\NormalTok{d  NaN    }\DecValTok{4}

\NormalTok{In [}\DecValTok{3}\NormalTok{]: pd.DataFrame(d, index}\OperatorTok{=}\NormalTok{[}\StringTok{"d"}\NormalTok{, }\StringTok{"b"}\NormalTok{, }\StringTok{"a"}\NormalTok{])}
\NormalTok{Out[}\DecValTok{3}\NormalTok{]: }
\NormalTok{   one  two}
\NormalTok{d  NaN    }\DecValTok{4}
\NormalTok{b  }\FloatTok{2.0}    \DecValTok{3}
\NormalTok{a  }\FloatTok{3.0}    \DecValTok{2}

\NormalTok{In [}\DecValTok{4}\NormalTok{]: pd.DataFrame(d, index}\OperatorTok{=}\NormalTok{[}\StringTok{"d"}\NormalTok{, }\StringTok{"b"}\NormalTok{, }\StringTok{"a"}\NormalTok{], columns}\OperatorTok{=}\NormalTok{[}\StringTok{"two"}\NormalTok{, }\StringTok{"three"}\NormalTok{])}
\NormalTok{Out[}\DecValTok{4}\NormalTok{]:}
\NormalTok{   two  three}
\NormalTok{d    }\DecValTok{4}\NormalTok{    NaN}
\NormalTok{b    }\DecValTok{3}\NormalTok{    NaN}
\NormalTok{a    }\DecValTok{2}\NormalTok{    NaN}
\end{Highlighting}
\end{Shaded}

\textbf{Khởi tạo DataFrame từ dict của numpy.ndarray/List}

Đối với việc khởi tạo này, bắt buộc các mảng phải có cùng độ dài. Khi không truyền \texttt{index} vào thì
index của DataFrame sẽ được tạo từ \texttt{0} đến \texttt{len(n)\ -\ 1} trong đó \texttt{n} là độ dài của mảng. Khi truyền
giá trị \texttt{columns}, DataFrame sẽ chọn những key thuộc dict và cũng thuộc \texttt{columns}, giá trị trong
\texttt{columns} không có trong key của dict sẽ được gán \texttt{NaN}

\begin{Shaded}
\begin{Highlighting}[]
\NormalTok{In [}\DecValTok{1}\NormalTok{]: d }\OperatorTok{=}\NormalTok{ \{}
            \StringTok{"one"}\NormalTok{: [}\DecValTok{1}\NormalTok{, }\DecValTok{2}\NormalTok{, }\DecValTok{3}\NormalTok{, }\DecValTok{4}\NormalTok{],}
            \StringTok{"two"}\NormalTok{: [}\DecValTok{1}\NormalTok{, }\DecValTok{2}\NormalTok{, }\DecValTok{3}\NormalTok{, }\DecValTok{4}\NormalTok{],}
            \StringTok{"three"}\NormalTok{: [}\DecValTok{1}\NormalTok{, }\DecValTok{2}\NormalTok{, }\DecValTok{3}\NormalTok{, }\DecValTok{4}\NormalTok{]}
\NormalTok{        \}}
\NormalTok{In [}\DecValTok{2}\NormalTok{]: pd.DataFrame(data}\OperatorTok{=}\NormalTok{d,}
\NormalTok{                     index}\OperatorTok{=}\NormalTok{[}\StringTok{"a"}\NormalTok{, }\StringTok{"b"}\NormalTok{, }\StringTok{"c"}\NormalTok{, }\StringTok{"d"}\NormalTok{],}
\NormalTok{                     columns}\OperatorTok{=}\NormalTok{[}\StringTok{"one"}\NormalTok{, }\StringTok{"two"}\NormalTok{, }\StringTok{"four"}\NormalTok{])}
\NormalTok{Out[}\DecValTok{2}\NormalTok{]:}
\NormalTok{   one  two four}
\NormalTok{a    }\DecValTok{1}    \DecValTok{1}\NormalTok{   NaN}
\NormalTok{b    }\DecValTok{2}    \DecValTok{2}\NormalTok{   NaN}
\NormalTok{c    }\DecValTok{3}    \DecValTok{3}\NormalTok{   NaN}
\NormalTok{d    }\DecValTok{4}    \DecValTok{4}\NormalTok{   NaN}
\end{Highlighting}
\end{Shaded}

\textbf{Khởi tạo DataFrame từ Mảng 2 chiều/ 2-d numpy.ndarray}

Khi không truyền \texttt{index} vào thì index của \texttt{DataFrame} sẽ được tạo từ \texttt{0} đến \texttt{len(n)\ -\ 1} trong đó \texttt{n}
là số lượng List con hoặc là số dòng hay \texttt{shape{[}0{]}} của \texttt{numpy.ndarray}. Khi không truyền \texttt{columns}
thì tên columns sẽ được tạo từ \texttt{0} đến \texttt{len(n)\ -\ 1} với \texttt{n} là độ dài lớn nhất của List con hoặc \texttt{shape{[}1{]}}
của \texttt{numpy.ndarray}

\begin{Shaded}
\begin{Highlighting}[]
\NormalTok{In [}\DecValTok{1}\NormalTok{]: pd.DataFrame(data}\OperatorTok{=}\NormalTok{[[}\DecValTok{1}\NormalTok{, }\DecValTok{2}\NormalTok{], [}\DecValTok{3}\NormalTok{, }\DecValTok{4}\NormalTok{, }\DecValTok{5}\NormalTok{]], }
\NormalTok{                     index}\OperatorTok{=}\NormalTok{[}\StringTok{"a"}\NormalTok{, }\StringTok{"b"}\NormalTok{], }
\NormalTok{                     columns}\OperatorTok{=}\NormalTok{[}\StringTok{\textquotesingle{}one\textquotesingle{}}\NormalTok{,}\StringTok{\textquotesingle{}two\textquotesingle{}}\NormalTok{,}\StringTok{\textquotesingle{}three\textquotesingle{}}\NormalTok{])}
\NormalTok{Out[}\DecValTok{1}\NormalTok{]: }
\NormalTok{   one  two  three}
\NormalTok{a    }\DecValTok{1}    \DecValTok{2}\NormalTok{    NaN}
\NormalTok{b    }\DecValTok{3}    \DecValTok{4}    \FloatTok{5.0}

\NormalTok{In [}\DecValTok{2}\NormalTok{]: pd.DataFrame(data}\OperatorTok{=}\NormalTok{np.random.rand(}\DecValTok{2}\NormalTok{,}\DecValTok{3}\NormalTok{), }
\NormalTok{                     index}\OperatorTok{=}\NormalTok{[}\StringTok{"a"}\NormalTok{, }\StringTok{"b"}\NormalTok{], }
\NormalTok{                     columns}\OperatorTok{=}\NormalTok{[}\StringTok{\textquotesingle{}one\textquotesingle{}}\NormalTok{,}\StringTok{\textquotesingle{}two\textquotesingle{}}\NormalTok{,}\StringTok{\textquotesingle{}three\textquotesingle{}}\NormalTok{]))}
\NormalTok{Out[}\DecValTok{2}\NormalTok{]:}
\NormalTok{        one       two     three}
\NormalTok{a  }\FloatTok{0.662008}  \FloatTok{0.085735}  \FloatTok{0.331281}
\NormalTok{b  }\FloatTok{0.115360}  \FloatTok{0.358092}  \FloatTok{0.862477}
\end{Highlighting}
\end{Shaded}

\textbf{Khởi tạo DataFrame từ danh sách các dict}

Ở cách khởi tạo này, bạn hãy tưởng tượng rằng mỗi dict là một dòng của DataFrame với các key là tên
cột và value là giá trị tại cột đó. Việc truyền thêm hoặc không truyền \texttt{index} cũng giống
như các trường hợp khởi tạo trên.

\begin{rmdnote}
\textbf{\emph{Lưu ý:}} Trong trường hợp này, nếu bạn truyền \texttt{columns} vào thì \texttt{columns} bắt buộc phải chứa tất cả
các key của dict
\end{rmdnote}

Trong ví dụ dưới đây, \texttt{columns} phải chứa toàn bộ keys \texttt{{[}"one",\ "two",\ "three"{]}}, nếu thiếu 1 trong 3
sẽ phát sinh lỗi.

\begin{Shaded}
\begin{Highlighting}[]
\NormalTok{In [}\DecValTok{1}\NormalTok{]: d }\OperatorTok{=}\NormalTok{ [\{}\StringTok{"one"}\NormalTok{: }\DecValTok{1}\NormalTok{, }\StringTok{"two"}\NormalTok{: }\DecValTok{2}\NormalTok{\}, \{}\StringTok{"one"}\NormalTok{: }\DecValTok{4}\NormalTok{, }\StringTok{"two"}\NormalTok{: }\DecValTok{5}\NormalTok{, }\StringTok{"three"}\NormalTok{: }\DecValTok{6}\NormalTok{\}]}
\NormalTok{In [}\DecValTok{2}\NormalTok{]: pd.DataFrame(d, index}\OperatorTok{=}\NormalTok{[}\StringTok{"a"}\NormalTok{, }\StringTok{"b"}\NormalTok{], columns}\OperatorTok{=}\NormalTok{[}\StringTok{"one"}\NormalTok{, }\StringTok{"two"}\NormalTok{, }\StringTok{"three"}\NormalTok{, }\StringTok{"four"}\NormalTok{])}
\NormalTok{Out[}\DecValTok{2}\NormalTok{]:}
\NormalTok{   one  two  three  four}
\NormalTok{a    }\DecValTok{1}    \DecValTok{2}\NormalTok{    NaN   NaN}
\NormalTok{b    }\DecValTok{4}    \DecValTok{5}    \FloatTok{6.0}\NormalTok{   NaN}
\end{Highlighting}
\end{Shaded}

\textbf{Khởi tạo DataFrame từ Mảng có cấu trúc}

Mảng có cấu trúc là mảng mà các phần tử của nó là một cấu trúc, bao gồm các thành phần nhỏ hơn, các thành phần này được đặt tên và khai báo kiểu dữ liệu.
Dưới đây là một ví dụ Mảng có cấu trúc trong numpy

\begin{Shaded}
\begin{Highlighting}[]
\NormalTok{In [}\DecValTok{1}\NormalTok{]: data }\OperatorTok{=}\NormalTok{ np.array([(}\StringTok{\textquotesingle{}pikachu\textquotesingle{}}\NormalTok{, }\DecValTok{9}\NormalTok{, }\FloatTok{27.0}\NormalTok{), (}\StringTok{\textquotesingle{}mewtwo\textquotesingle{}}\NormalTok{, }\DecValTok{3}\NormalTok{, }\FloatTok{81.0}\NormalTok{)],}
\NormalTok{                        dtype}\OperatorTok{=}\NormalTok{[(}\StringTok{\textquotesingle{}name\textquotesingle{}}\NormalTok{, }\StringTok{\textquotesingle{}U10\textquotesingle{}}\NormalTok{), (}\StringTok{\textquotesingle{}age\textquotesingle{}}\NormalTok{, }\StringTok{\textquotesingle{}i4\textquotesingle{}}\NormalTok{), (}\StringTok{\textquotesingle{}weight\textquotesingle{}}\NormalTok{, }\StringTok{\textquotesingle{}f4\textquotesingle{}}\NormalTok{)])}
\NormalTok{In [}\DecValTok{2}\NormalTok{]: pd.DataFrame(data)}
\NormalTok{Out[}\DecValTok{2}\NormalTok{]: }
\NormalTok{       name  age  weight}
\DecValTok{0}\NormalTok{   pikachu    }\DecValTok{9}    \FloatTok{27.0}
\DecValTok{1}\NormalTok{    mewtwo    }\DecValTok{3}    \FloatTok{81.0}
\end{Highlighting}
\end{Shaded}

\textbf{Khởi tạo DataFrame từ namedtuple}

Các trường trong \texttt{nametuple} sẽ được gán thành tên các columns trong \texttt{DataFrame}. Những giá trị của \texttt{namedtuple} sẽ được xem là 1 dòng trong \texttt{DataFrame}.
Số lượng cột của \texttt{DataFrame} sẽ phụ thuộc vào số lượng giá trị của phần từ \texttt{namedtuple} đầu tiên. Nếu các phần tử phía sau có số lượng giá trị ít hơn thì
sẽ được điền \texttt{NaN} và ngược lại sẽ trả ra lỗi nếu số lượng giá trị của \texttt{namedtuple} lớn hơn số lượng giá trị của phần tử \texttt{namedtuple} đầu tiên.

Ví dụ về cách tạo namedtuple

\begin{Shaded}
\begin{Highlighting}[]
\ImportTok{from}\NormalTok{ collections }\ImportTok{import}\NormalTok{ namedtuple}
\NormalTok{Point2D }\OperatorTok{=}\NormalTok{ namedtuple(}\StringTok{"Point2D"}\NormalTok{, }\StringTok{"x y"}\NormalTok{)}
\NormalTok{Point3D }\OperatorTok{=}\NormalTok{ namedtuple(}\StringTok{"Point3D"}\NormalTok{, }\StringTok{"x y z"}\NormalTok{)}
\end{Highlighting}
\end{Shaded}

Tạo DataFrame từ namedtuple \texttt{Point2D}

\begin{Shaded}
\begin{Highlighting}[]
\NormalTok{In [}\DecValTok{1}\NormalTok{]: pd.DataFrame([Point2D(}\DecValTok{0}\NormalTok{, }\DecValTok{0}\NormalTok{), Point2D(}\DecValTok{0}\NormalTok{, }\DecValTok{1}\NormalTok{), Point2D(}\DecValTok{0}\NormalTok{, }\DecValTok{2}\NormalTok{)])}
\NormalTok{Out[}\DecValTok{1}\NormalTok{]:}
\NormalTok{   x  y}
\DecValTok{0}  \DecValTok{0}  \DecValTok{0}
\DecValTok{1}  \DecValTok{0}  \DecValTok{1}
\DecValTok{2}  \DecValTok{0}  \DecValTok{2}
\end{Highlighting}
\end{Shaded}

Tạo DataFrame từ namedtuple cả \texttt{Point2D} và \texttt{Point3D}

\begin{Shaded}
\begin{Highlighting}[]
\NormalTok{In [}\DecValTok{1}\NormalTok{]: pd.DataFrame([Point3D(}\DecValTok{0}\NormalTok{, }\DecValTok{0}\NormalTok{, }\DecValTok{0}\NormalTok{), Point2D(}\DecValTok{0}\NormalTok{, }\DecValTok{1}\NormalTok{), Point3D(}\DecValTok{0}\NormalTok{, }\DecValTok{2}\NormalTok{, }\DecValTok{3}\NormalTok{)])}
\NormalTok{Out[}\DecValTok{1}\NormalTok{]:    }
\NormalTok{   x  y    z}
\DecValTok{0}  \DecValTok{0}  \DecValTok{0}  \FloatTok{0.0}
\DecValTok{1}  \DecValTok{0}  \DecValTok{1}\NormalTok{  NaN}
\DecValTok{2}  \DecValTok{0}  \DecValTok{2}  \FloatTok{3.0}
\end{Highlighting}
\end{Shaded}

Như ta thấy, tại phần tử thứ 2 chỉ có 2 giá trị, trong khi phần tử thứ nhất có 3 giá trị, vậy nên phần tử bị thiếu tại cột \texttt{z} sẽ được gán \texttt{NaN}

\textbf{Khởi tạo DataFrame từ Series}

\begin{Shaded}
\begin{Highlighting}[]
\NormalTok{In [}\DecValTok{1}\NormalTok{]: s }\OperatorTok{=}\NormalTok{ pd.Series(data}\OperatorTok{=}\NormalTok{[}\DecValTok{0}\NormalTok{, }\DecValTok{1}\NormalTok{, }\DecValTok{2}\NormalTok{], index}\OperatorTok{=}\NormalTok{[}\StringTok{"a"}\NormalTok{, }\StringTok{"b"}\NormalTok{, }\StringTok{"c"}\NormalTok{], name}\OperatorTok{=}\StringTok{"meow"}\NormalTok{)}
\NormalTok{In [}\DecValTok{2}\NormalTok{]: pd.DataFrame(s)}
\NormalTok{Out[}\DecValTok{2}\NormalTok{]: }
\NormalTok{   meow}
\NormalTok{a     }\DecValTok{0}
\NormalTok{b     }\DecValTok{1}
\NormalTok{c     }\DecValTok{2}
\end{Highlighting}
\end{Shaded}

\texttt{name} của Series sẽ là tên cột của DataFrame và \texttt{index} của Series sẽ là index của DataFrame nếu ta không truyền các biến \texttt{index}, \texttt{columns} khi khởi tạo \texttt{pd.DataFrame}

\hypertarget{cuxe1c-huxe0m-khux1edfi-tux1ea1o-thay-thux1ebf}{%
\subsection{Các hàm khởi tạo thay thế}\label{cuxe1c-huxe0m-khux1edfi-tux1ea1o-thay-thux1ebf}}

\textbf{DataFrame.from\_dict}

Cách khởi tạo

\begin{Shaded}
\begin{Highlighting}[]
\NormalTok{pd.DataFrame.from\_dict(data, orient}\OperatorTok{=}\StringTok{\textquotesingle{}columns\textquotesingle{}}\NormalTok{, dtype}\OperatorTok{=}\VariableTok{None}\NormalTok{, columns}\OperatorTok{=}\VariableTok{None}\NormalTok{)}
\end{Highlighting}
\end{Shaded}

\texttt{data} truyền vào là 1 dict, \texttt{orient} có 2 giá trị có thể đưa vào là \texttt{\{"columns",\ "index"\}}, \texttt{columns} là danh sách tên các cột của DataFrame.

\begin{rmdnote}
\textbf{\emph{Lưu ý:}} Chỉ được truyền \texttt{columns} khi \texttt{orient="index"}. Khi \texttt{orient="columns"} sẽ báo lỗi.
\end{rmdnote}

Ví dụ tạo DataFrame khi \texttt{orient="columns"}. Với cách khởi tạo này tên các cột của DataFrame sẽ là key của dict

\begin{Shaded}
\begin{Highlighting}[]
\NormalTok{In [}\DecValTok{1}\NormalTok{]: data }\OperatorTok{=}\NormalTok{ \{}\StringTok{"col\_1"}\NormalTok{: [}\DecValTok{3}\NormalTok{, }\DecValTok{2}\NormalTok{, }\DecValTok{1}\NormalTok{, }\DecValTok{0}\NormalTok{], }\StringTok{"col\_2"}\NormalTok{: [}\StringTok{"a"}\NormalTok{, }\StringTok{"b"}\NormalTok{, }\StringTok{"c"}\NormalTok{, }\StringTok{"d"}\NormalTok{]\}}
\NormalTok{In [}\DecValTok{2}\NormalTok{]: pd.DataFrame.from\_dict(data)}
\NormalTok{Out[}\DecValTok{2}\NormalTok{]:}
\NormalTok{   col\_1 col\_2}
\DecValTok{0}      \DecValTok{3}\NormalTok{     a}
\DecValTok{1}      \DecValTok{2}\NormalTok{     b}
\DecValTok{2}      \DecValTok{1}\NormalTok{     c}
\DecValTok{3}      \DecValTok{0}\NormalTok{     d}
\end{Highlighting}
\end{Shaded}

Ví dụ tạo DataFrame khi \texttt{orient="index"}. Với cách khởi tạo này index của DataFrame sẽ là key của dict.

\begin{Shaded}
\begin{Highlighting}[]
\NormalTok{In [}\DecValTok{1}\NormalTok{]: data }\OperatorTok{=}\NormalTok{ \{}\StringTok{"col\_1"}\NormalTok{: [}\DecValTok{3}\NormalTok{, }\DecValTok{2}\NormalTok{, }\DecValTok{1}\NormalTok{, }\DecValTok{0}\NormalTok{], }\StringTok{"col\_2"}\NormalTok{: [}\StringTok{"a"}\NormalTok{, }\StringTok{"b"}\NormalTok{, }\StringTok{"c"}\NormalTok{, }\StringTok{"d"}\NormalTok{]\}}
\NormalTok{In [}\DecValTok{2}\NormalTok{]: pd.DataFrame.from\_dict(data, orient}\OperatorTok{=}\StringTok{"index"}\NormalTok{, }
\NormalTok{                               columns}\OperatorTok{=}\NormalTok{[}\StringTok{"one"}\NormalTok{, }\StringTok{"two"}\NormalTok{, }\StringTok{"three"}\NormalTok{, }\StringTok{"four"}\NormalTok{])}
\NormalTok{Out[}\DecValTok{2}\NormalTok{]:}
\NormalTok{   col\_1 col\_2}
\DecValTok{0}      \DecValTok{3}\NormalTok{     a}
\DecValTok{1}      \DecValTok{2}\NormalTok{     b}
\DecValTok{2}      \DecValTok{1}\NormalTok{     c}
\DecValTok{3}      \DecValTok{0}\NormalTok{     d}
\end{Highlighting}
\end{Shaded}

\textbf{DataFrame.from\_records}

Cách khởi tạo

\begin{Shaded}
\begin{Highlighting}[]
\NormalTok{pd.DataFrame.from\_records(data)}
\end{Highlighting}
\end{Shaded}

\texttt{data} truyền vào có thể là một mảng có cấu trúc

\begin{Shaded}
\begin{Highlighting}[]
\NormalTok{In [}\DecValTok{1}\NormalTok{]: data }\OperatorTok{=}\NormalTok{ np.array([(}\StringTok{\textquotesingle{}Rex\textquotesingle{}}\NormalTok{, }\DecValTok{9}\NormalTok{, }\FloatTok{81.0}\NormalTok{), (}\StringTok{\textquotesingle{}Fido\textquotesingle{}}\NormalTok{, }\DecValTok{3}\NormalTok{, }\FloatTok{27.0}\NormalTok{)],}
\NormalTok{                        dtype}\OperatorTok{=}\NormalTok{[(}\StringTok{\textquotesingle{}name\textquotesingle{}}\NormalTok{, }\StringTok{\textquotesingle{}U10\textquotesingle{}}\NormalTok{), (}\StringTok{\textquotesingle{}age\textquotesingle{}}\NormalTok{, }\StringTok{\textquotesingle{}i4\textquotesingle{}}\NormalTok{), (}\StringTok{\textquotesingle{}weight\textquotesingle{}}\NormalTok{, }\StringTok{\textquotesingle{}f4\textquotesingle{}}\NormalTok{)])}
\NormalTok{In [}\DecValTok{2}\NormalTok{]: pd.DataFrame.from\_records(data, index}\OperatorTok{=}\NormalTok{[}\StringTok{"a"}\NormalTok{, }\StringTok{"b"}\NormalTok{])}
\NormalTok{Out[}\DecValTok{2}\NormalTok{]: }
\NormalTok{   name  age  weight}
\NormalTok{a   Rex    }\DecValTok{9}    \FloatTok{81.0}
\NormalTok{b  Fido    }\DecValTok{3}    \FloatTok{27.0}
\end{Highlighting}
\end{Shaded}

Dữ liệu có thể một danh sách các namedtuple

\begin{Shaded}
\begin{Highlighting}[]
\ImportTok{from}\NormalTok{ collections }\ImportTok{import}\NormalTok{ namedtuple}
\NormalTok{Point2D }\OperatorTok{=}\NormalTok{ namedtuple(}\StringTok{"Point2D"}\NormalTok{, }\StringTok{"x y"}\NormalTok{)}
\NormalTok{Point3D }\OperatorTok{=}\NormalTok{ namedtuple(}\StringTok{"Point3D"}\NormalTok{, }\StringTok{"x y z"}\NormalTok{)}
\NormalTok{pd.DataFrame.from\_records([Point3D(}\DecValTok{0}\NormalTok{, }\DecValTok{0}\NormalTok{, }\DecValTok{0}\NormalTok{), Point2D(}\DecValTok{0}\NormalTok{, }\DecValTok{1}\NormalTok{), Point3D(}\DecValTok{0}\NormalTok{, }\DecValTok{2}\NormalTok{, }\DecValTok{3}\NormalTok{)],}
\NormalTok{                          columns}\OperatorTok{=}\NormalTok{[}\StringTok{"x"}\NormalTok{,}\StringTok{"y"}\NormalTok{,}\StringTok{"z"}\NormalTok{], index}\OperatorTok{=}\NormalTok{[}\StringTok{"a"}\NormalTok{, }\StringTok{"b"}\NormalTok{, }\StringTok{"c"}\NormalTok{])}
\end{Highlighting}
\end{Shaded}

\begin{Shaded}
\begin{Highlighting}[]
\NormalTok{   x  y    z}
\NormalTok{a  }\DecValTok{0}  \DecValTok{0}  \FloatTok{0.0}
\NormalTok{b  }\DecValTok{0}  \DecValTok{1}\NormalTok{  NaN}
\NormalTok{c  }\DecValTok{0}  \DecValTok{2}  \FloatTok{3.0}
\end{Highlighting}
\end{Shaded}

Hoặc 1 danh sách các dict

\begin{Shaded}
\begin{Highlighting}[]
\NormalTok{In [}\DecValTok{1}\NormalTok{]: d }\OperatorTok{=}\NormalTok{ [\{}\StringTok{"one"}\NormalTok{: }\DecValTok{1}\NormalTok{, }\StringTok{"two"}\NormalTok{: }\DecValTok{2}\NormalTok{\}, \{}\StringTok{"one"}\NormalTok{: }\DecValTok{4}\NormalTok{, }\StringTok{"two"}\NormalTok{: }\DecValTok{5}\NormalTok{, }\StringTok{"three"}\NormalTok{: }\DecValTok{6}\NormalTok{\}]}
\NormalTok{In [}\DecValTok{2}\NormalTok{]: pd.DataFrame.from\_records(d, index}\OperatorTok{=}\NormalTok{[}\StringTok{"a"}\NormalTok{, }\StringTok{"b"}\NormalTok{], columns}\OperatorTok{=}\NormalTok{[}\StringTok{"one"}\NormalTok{, }\StringTok{"two"}\NormalTok{, }\StringTok{"three"}\NormalTok{, }\StringTok{"four"}\NormalTok{])}
\NormalTok{Out[}\DecValTok{2}\NormalTok{]:}
\NormalTok{   one  two  three  four}
\NormalTok{a    }\DecValTok{1}    \DecValTok{2}\NormalTok{    NaN   NaN}
\NormalTok{b    }\DecValTok{4}    \DecValTok{5}    \FloatTok{6.0}\NormalTok{   NaN}
\end{Highlighting}
\end{Shaded}

\hypertarget{data-type-trong-pandas}{%
\section{Data type trong pandas}\label{data-type-trong-pandas}}

Để kiểm tra kiểu dữ liệu của \texttt{Series} hay \texttt{DataFrame} bạn có thể gọi thuộc tính \texttt{dtypes}. Các kiểu dữ liệu thường gặp của Pandas được mô tả theo bảng dưới đây:

\begin{longtable}[]{@{}
  >{\raggedright\arraybackslash}p{(\columnwidth - 4\tabcolsep) * \real{0.33}}
  >{\raggedright\arraybackslash}p{(\columnwidth - 4\tabcolsep) * \real{0.25}}
  >{\raggedright\arraybackslash}p{(\columnwidth - 4\tabcolsep) * \real{0.21}}@{}}
\toprule
Các kiểu dữ liệu
phổ biến & Numpy/Pandas
object & Hiển thị \\
\midrule
\endhead
Boolean & np.bool & \emph{bool} \\
Integer & np.int,
np.uint & \emph{int}
\emph{uint} \\
Float & np.float & \emph{float} \\
Object & np.object & \emph{O, object} \\
Datetime & np.datetime64,
pd.Timestamp & \emph{datetime64} \\
Timedelta & np.timedelta64,
pd.Timedelta & \emph{timedelta64} \\
Category & pd.Categorical & \emph{category} \\
Complex & np.complex & \emph{complex} \\
\bottomrule
\end{longtable}

Ví dụ

\begin{Shaded}
\begin{Highlighting}[]
\NormalTok{In [}\DecValTok{1}\NormalTok{]: df }\OperatorTok{=}\NormalTok{ pd.DataFrame(\{}
                           \StringTok{\textquotesingle{}col\_1\textquotesingle{}}\NormalTok{: [}\DecValTok{1}\NormalTok{, }\DecValTok{0}\NormalTok{, }\DecValTok{1}\NormalTok{, }\DecValTok{0}\NormalTok{], }
                           \StringTok{\textquotesingle{}col\_2\textquotesingle{}}\NormalTok{: [}\FloatTok{1.0}\NormalTok{, }\FloatTok{2.0}\NormalTok{, }\FloatTok{3.0}\NormalTok{, }\FloatTok{4.0}\NormalTok{], }
                           \StringTok{\textquotesingle{}col\_3\textquotesingle{}}\NormalTok{: [}\StringTok{\textquotesingle{}1\textquotesingle{}}\NormalTok{, }\StringTok{\textquotesingle{}2\textquotesingle{}}\NormalTok{, }\StringTok{\textquotesingle{}3\textquotesingle{}}\NormalTok{, }\StringTok{\textquotesingle{}4\textquotesingle{}}\NormalTok{],}
                           \StringTok{\textquotesingle{}col\_4\textquotesingle{}}\NormalTok{: [}\StringTok{\textquotesingle{}1\textquotesingle{}}\NormalTok{, }\DecValTok{2}\NormalTok{, }\StringTok{\textquotesingle{}3\textquotesingle{}}\NormalTok{, }\DecValTok{4}\NormalTok{],}
                           \StringTok{\textquotesingle{}col\_5\textquotesingle{}}\NormalTok{: [}\VariableTok{True}\NormalTok{, }\VariableTok{False}\NormalTok{, }\VariableTok{True}\NormalTok{, }\VariableTok{False}\NormalTok{]\})}
\NormalTok{In [}\DecValTok{2}\NormalTok{]: df}
\NormalTok{Out[}\DecValTok{2}\NormalTok{]:}
\NormalTok{   col\_1  col\_2 col\_3 col\_4  col\_5}
\DecValTok{0}      \DecValTok{1}    \FloatTok{1.0}     \DecValTok{1}     \DecValTok{1}   \VariableTok{True}
\DecValTok{1}      \DecValTok{0}    \FloatTok{2.0}     \DecValTok{2}     \DecValTok{2}  \VariableTok{False}
\DecValTok{2}      \DecValTok{1}    \FloatTok{3.0}     \DecValTok{3}     \DecValTok{3}   \VariableTok{True}
\DecValTok{3}      \DecValTok{0}    \FloatTok{4.0}     \DecValTok{4}     \DecValTok{4}  \VariableTok{False}
\end{Highlighting}
\end{Shaded}

\begin{Shaded}
\begin{Highlighting}[]
\NormalTok{In [}\DecValTok{3}\NormalTok{]: df.dtypes}
\NormalTok{Out[}\DecValTok{3}\NormalTok{]:}
\NormalTok{col\_1      int64}
\NormalTok{col\_2    float64}
\NormalTok{col\_3     }\BuiltInTok{object}
\NormalTok{col\_4     }\BuiltInTok{object}
\NormalTok{col\_5       }\BuiltInTok{bool}
\NormalTok{dtype: }\BuiltInTok{object}
\end{Highlighting}
\end{Shaded}

Nếu không khai báo kiểu dữ liệu khi khởi tạo, pandas sẽ mặc định kiểu dữ liệu là \texttt{int64}, \texttt{float64} và \texttt{object} và \texttt{bool}.

\hypertarget{nhux1eadp-xuux1ea5t-trong-pandas}{%
\chapter{Nhập xuất trong pandas}\label{nhux1eadp-xuux1ea5t-trong-pandas}}

\hypertarget{ux111ux1ecdc-vuxe0-lux1b0u-file}{%
\section{Đọc và lưu file}\label{ux111ux1ecdc-vuxe0-lux1b0u-file}}

\hypertarget{cux1ea5u-huxecnh-pandas}{%
\section{Cấu hình pandas}\label{cux1ea5u-huxecnh-pandas}}

\hypertarget{mux1ed9t-sux1ed1-huxe0m-cux1a1-bux1ea3n}{%
\chapter{Một số hàm cơ bản}\label{mux1ed9t-sux1ed1-huxe0m-cux1a1-bux1ea3n}}

\hypertarget{lux1eb7p-trong-pandas}{%
\chapter{Lặp trong Pandas}\label{lux1eb7p-trong-pandas}}

\hypertarget{sux1eed-dux1ee5ng-vectorizer}{%
\section{Sử dụng vectorizer}\label{sux1eed-dux1ee5ng-vectorizer}}

\hypertarget{sux1eed-dux1ee5ng-apply}{%
\section{Sử dụng apply}\label{sux1eed-dux1ee5ng-apply}}

\hypertarget{sux1eed-dux1ee5ng-iterator}{%
\section{Sử dụng iterator}\label{sux1eed-dux1ee5ng-iterator}}

\hypertarget{xux1eed-luxfd-song-song-trong-pandas}{%
\section{Xử lý song song trong pandas}\label{xux1eed-luxfd-song-song-trong-pandas}}

\hypertarget{select-vuxe0-filter}{%
\chapter{Select và Filter}\label{select-vuxe0-filter}}

\hypertarget{index}{%
\section{Index}\label{index}}

\hypertarget{loc-vuxe0-iloc}{%
\section{loc và iloc}\label{loc-vuxe0-iloc}}

\hypertarget{lux1ecdc-theo-ux111iux1ec1u-kiux1ec7n}{%
\section{Lọc theo điều kiện}\label{lux1ecdc-theo-ux111iux1ec1u-kiux1ec7n}}

\hypertarget{cuxe1c-cuxe1ch-phux1ed1i-hux1ee3p-nhiux1ec1u-bux1ea3ng-vux1edbi-nhau}{%
\chapter{Các cách phối hợp nhiều bảng với nhau}\label{cuxe1c-cuxe1ch-phux1ed1i-hux1ee3p-nhiux1ec1u-bux1ea3ng-vux1edbi-nhau}}

\hypertarget{join}{%
\section{Join}\label{join}}

\hypertarget{merge}{%
\section{Merge}\label{merge}}

\hypertarget{concat}{%
\section{Concat}\label{concat}}

\hypertarget{groupby-vuxe0-aggregate}{%
\chapter{Groupby và Aggregate}\label{groupby-vuxe0-aggregate}}

\hypertarget{luxe0m-viux1ec7c-vux1edbi-1-sux1ed1-kiux1ec3u-dux1eef-liux1ec7u}{%
\chapter{Làm việc với 1 số kiểu dữ liệu}\label{luxe0m-viux1ec7c-vux1edbi-1-sux1ed1-kiux1ec3u-dux1eef-liux1ec7u}}

\hypertarget{xux1eed-luxfd-dux1eef-liux1ec7u-dux1ea1ng-text}{%
\section{Xử lý dữ liệu dạng text}\label{xux1eed-luxfd-dux1eef-liux1ec7u-dux1ea1ng-text}}

\hypertarget{xux1eed-luxfd-dux1eef-liux1ec7u-dux1ea1ng-timestamp}{%
\section{Xử lý dữ liệu dạng timestamp}\label{xux1eed-luxfd-dux1eef-liux1ec7u-dux1ea1ng-timestamp}}

\hypertarget{category-trong-pandas}{%
\section{Category trong pandas}\label{category-trong-pandas}}

\hypertarget{xux1eed-luxfd-missing-data}{%
\section{Xử lý Missing data}\label{xux1eed-luxfd-missing-data}}

\hypertarget{mux1ed9t-sux1ed1-kiux1ebfn-thux1ee9c-nuxe2ng-cao}{%
\chapter{Một số kiến thức nâng cao}\label{mux1ed9t-sux1ed1-kiux1ebfn-thux1ee9c-nuxe2ng-cao}}

\hypertarget{multiindex}{%
\section{MultiIndex}\label{multiindex}}

\hypertarget{pivot-vuxe0-merge}{%
\section{Pivot và Merge}\label{pivot-vuxe0-merge}}

\hypertarget{resample}{%
\section{Resample}\label{resample}}

\hypertarget{window}{%
\section{Window}\label{window}}

\hypertarget{anomaly-detection-project}{%
\chapter{Anomaly Detection Project}\label{anomaly-detection-project}}

\hypertarget{visualize-vux1edbi-matplotlib}{%
\chapter{Visualize với Matplotlib}\label{visualize-vux1edbi-matplotlib}}

\printindex

\end{document}
